\documentclass[10,portrait]{article}
\usepackage{lmodern}
\usepackage{amssymb,amsmath}
\usepackage{ifxetex,ifluatex}
\usepackage{fixltx2e} % provides \textsubscript
\ifnum 0\ifxetex 1\fi\ifluatex 1\fi=0 % if pdftex
  \usepackage[T1]{fontenc}
  \usepackage[utf8]{inputenc}
\else % if luatex or xelatex
  \ifxetex
    \usepackage{mathspec}
  \else
    \usepackage{fontspec}
  \fi
  \defaultfontfeatures{Ligatures=TeX,Scale=MatchLowercase}
\fi
% use upquote if available, for straight quotes in verbatim environments
\IfFileExists{upquote.sty}{\usepackage{upquote}}{}
% use microtype if available
\IfFileExists{microtype.sty}{%
\usepackage[]{microtype}
\UseMicrotypeSet[protrusion]{basicmath} % disable protrusion for tt fonts
}{}
\PassOptionsToPackage{hyphens}{url} % url is loaded by hyperref
\usepackage[unicode=true]{hyperref}
\PassOptionsToPackage{usenames,dvipsnames}{color} % color is loaded by hyperref
\hypersetup{
            pdftitle={R is dope AF},
            colorlinks=true,
            linkcolor=blue,
            citecolor=red,
            urlcolor=blue,
            breaklinks=true}
\urlstyle{same}  % don't use monospace font for urls
\usepackage[margin=1in]{geometry}
\usepackage[]{biblatex}
\usepackage{color}
\usepackage{fancyvrb}
\newcommand{\VerbBar}{|}
\newcommand{\VERB}{\Verb[commandchars=\\\{\}]}
\DefineVerbatimEnvironment{Highlighting}{Verbatim}{commandchars=\\\{\}}
% Add ',fontsize=\small' for more characters per line
\usepackage{framed}
\definecolor{shadecolor}{RGB}{248,248,248}
\newenvironment{Shaded}{\begin{snugshade}}{\end{snugshade}}
\newcommand{\KeywordTok}[1]{\textcolor[rgb]{0.13,0.29,0.53}{\textbf{#1}}}
\newcommand{\DataTypeTok}[1]{\textcolor[rgb]{0.13,0.29,0.53}{#1}}
\newcommand{\DecValTok}[1]{\textcolor[rgb]{0.00,0.00,0.81}{#1}}
\newcommand{\BaseNTok}[1]{\textcolor[rgb]{0.00,0.00,0.81}{#1}}
\newcommand{\FloatTok}[1]{\textcolor[rgb]{0.00,0.00,0.81}{#1}}
\newcommand{\ConstantTok}[1]{\textcolor[rgb]{0.00,0.00,0.00}{#1}}
\newcommand{\CharTok}[1]{\textcolor[rgb]{0.31,0.60,0.02}{#1}}
\newcommand{\SpecialCharTok}[1]{\textcolor[rgb]{0.00,0.00,0.00}{#1}}
\newcommand{\StringTok}[1]{\textcolor[rgb]{0.31,0.60,0.02}{#1}}
\newcommand{\VerbatimStringTok}[1]{\textcolor[rgb]{0.31,0.60,0.02}{#1}}
\newcommand{\SpecialStringTok}[1]{\textcolor[rgb]{0.31,0.60,0.02}{#1}}
\newcommand{\ImportTok}[1]{#1}
\newcommand{\CommentTok}[1]{\textcolor[rgb]{0.56,0.35,0.01}{\textit{#1}}}
\newcommand{\DocumentationTok}[1]{\textcolor[rgb]{0.56,0.35,0.01}{\textbf{\textit{#1}}}}
\newcommand{\AnnotationTok}[1]{\textcolor[rgb]{0.56,0.35,0.01}{\textbf{\textit{#1}}}}
\newcommand{\CommentVarTok}[1]{\textcolor[rgb]{0.56,0.35,0.01}{\textbf{\textit{#1}}}}
\newcommand{\OtherTok}[1]{\textcolor[rgb]{0.56,0.35,0.01}{#1}}
\newcommand{\FunctionTok}[1]{\textcolor[rgb]{0.00,0.00,0.00}{#1}}
\newcommand{\VariableTok}[1]{\textcolor[rgb]{0.00,0.00,0.00}{#1}}
\newcommand{\ControlFlowTok}[1]{\textcolor[rgb]{0.13,0.29,0.53}{\textbf{#1}}}
\newcommand{\OperatorTok}[1]{\textcolor[rgb]{0.81,0.36,0.00}{\textbf{#1}}}
\newcommand{\BuiltInTok}[1]{#1}
\newcommand{\ExtensionTok}[1]{#1}
\newcommand{\PreprocessorTok}[1]{\textcolor[rgb]{0.56,0.35,0.01}{\textit{#1}}}
\newcommand{\AttributeTok}[1]{\textcolor[rgb]{0.77,0.63,0.00}{#1}}
\newcommand{\RegionMarkerTok}[1]{#1}
\newcommand{\InformationTok}[1]{\textcolor[rgb]{0.56,0.35,0.01}{\textbf{\textit{#1}}}}
\newcommand{\WarningTok}[1]{\textcolor[rgb]{0.56,0.35,0.01}{\textbf{\textit{#1}}}}
\newcommand{\AlertTok}[1]{\textcolor[rgb]{0.94,0.16,0.16}{#1}}
\newcommand{\ErrorTok}[1]{\textcolor[rgb]{0.64,0.00,0.00}{\textbf{#1}}}
\newcommand{\NormalTok}[1]{#1}
\usepackage{graphicx,grffile}
\makeatletter
\def\maxwidth{\ifdim\Gin@nat@width>\linewidth\linewidth\else\Gin@nat@width\fi}
\def\maxheight{\ifdim\Gin@nat@height>\textheight\textheight\else\Gin@nat@height\fi}
\makeatother
% Scale images if necessary, so that they will not overflow the page
% margins by default, and it is still possible to overwrite the defaults
% using explicit options in \includegraphics[width, height, ...]{}
\setkeys{Gin}{width=\maxwidth,height=\maxheight,keepaspectratio}
\IfFileExists{parskip.sty}{%
\usepackage{parskip}
}{% else
\setlength{\parindent}{0pt}
\setlength{\parskip}{6pt plus 2pt minus 1pt}
}
\setlength{\emergencystretch}{3em}  % prevent overfull lines
\providecommand{\tightlist}{%
  \setlength{\itemsep}{0pt}\setlength{\parskip}{0pt}}
\setcounter{secnumdepth}{0}
% Redefines (sub)paragraphs to behave more like sections
\ifx\paragraph\undefined\else
\let\oldparagraph\paragraph
\renewcommand{\paragraph}[1]{\oldparagraph{#1}\mbox{}}
\fi
\ifx\subparagraph\undefined\else
\let\oldsubparagraph\subparagraph
\renewcommand{\subparagraph}[1]{\oldsubparagraph{#1}\mbox{}}
\fi

% set default figure placement to htbp
\makeatletter
\def\fps@figure{htbp}
\makeatother


\title{R is dope AF}
\author{Matthew
Malishev\textsuperscript{1}*\\[2\baselineskip]\emph{\textsuperscript{1}
Department of Biology, Emory University, 1510 Clifton Road NE, Atlanta,
GA, USA, 30322}}
\date{}

\begin{document}
\maketitle

{
\hypersetup{linkcolor=black}
\setcounter{tocdepth}{4}
\tableofcontents
}
\newpage   

Date: 2018-11-16\\
R version: 3.5.0\\
*Corresponding author:
\href{mailto:matthew.malishev@gmail.com}{\nolinkurl{matthew.malishev@gmail.com}}\\
This document can be found at
\url{https://github.com/darwinanddavis/SchistoIBM/tree/master/mac}

~

R session info

\begin{Shaded}
\begin{Highlighting}[]
\NormalTok{params}\OperatorTok{$}\NormalTok{session}
\end{Highlighting}
\end{Shaded}

\begin{verbatim}
R version 3.5.0 (2018-04-23)
Platform: x86_64-apple-darwin15.6.0 (64-bit)
Running under: OS X El Capitan 10.11.6

Matrix products: default
BLAS: /Library/Frameworks/R.framework/Versions/3.5/Resources/lib/libRblas.0.dylib
LAPACK: /Library/Frameworks/R.framework/Versions/3.5/Resources/lib/libRlapack.dylib

locale:
[1] en_US.UTF-8/en_US.UTF-8/en_US.UTF-8/C/en_US.UTF-8/en_US.UTF-8

attached base packages:
[1] stats     graphics  grDevices utils     datasets  methods   base     

loaded via a namespace (and not attached):
 [1] compiler_3.5.0  backports_1.1.2 magrittr_1.5    rprojroot_1.3-2 tools_3.5.0     htmltools_0.3.6
 [7] pillar_1.2.3    tibble_1.4.2    yaml_2.2.0      Rcpp_0.12.19    stringi_1.2.3   rmarkdown_1.10 
[13] knitr_1.20      stringr_1.3.1   digest_0.6.15   rlang_0.3.0.1   evaluate_0.10.1
\end{verbatim}

\newpage  

\subsection{Overview}\label{overview}

This document showcases why \texttt{R} is \textbf{dope}.

You can write in-line \texttt{code}, equations like this \(E = mc^{2}\),
create links to your \href{https://github.com/darwinanddavis}{website}.

\subsubsection{\texorpdfstring{Just like LateX, but \emph{more
versatile}.}{Just like LateX, but more versatile.}}\label{just-like-latex-but-more-versatile.}

Create quoted text

\begin{quote}
Pump the bass in the trunk //\\
It rattled like a baby hand //\\
Except this toy cost 80 grand //\\
And I'm crazy tan, from all the places that I've been //\\
Just from writing words with a pen //
\end{quote}

Define equations\\
\[
t'=
\frac
{1}
{
\sqrt{1 - \frac
{v^{2}}
{c^{2}}
}
}
\]

Embed images/gifs:

\includegraphics{billnye.png}.

\newpage  

Embed code from different languages.

This is \texttt{R} code

\begin{Shaded}
\begin{Highlighting}[]
\ControlFlowTok{if}\NormalTok{(pck}\OperatorTok{==}\DecValTok{1}\NormalTok{)\{}
\NormalTok{  p<-}\KeywordTok{c}\NormalTok{(}\StringTok{"rJava"}\NormalTok{, }\StringTok{"RNetLogo"}\NormalTok{); }\KeywordTok{remove.packages}\NormalTok{(p)}
  \CommentTok{# then install rJava and RNetLogo from source}
  \KeywordTok{install.packages}\NormalTok{(}\StringTok{"rJava"}\NormalTok{, }\DataTypeTok{repos =} \StringTok{"https://cran.r-project.org/"}\NormalTok{, }\DataTypeTok{type=}\StringTok{"source"}\NormalTok{); }\KeywordTok{library}\NormalTok{(rJava)}
  \KeywordTok{install.packages}\NormalTok{(}\StringTok{"RNetLogo"}\NormalTok{, }\DataTypeTok{repos =} \StringTok{"https://cran.r-project.org/"}\NormalTok{, }\DataTypeTok{type=}\StringTok{"source"}\NormalTok{); }\KeywordTok{library}\NormalTok{(RNetLogo)}
\NormalTok{\}}
\end{Highlighting}
\end{Shaded}

\texttt{shell/bash}

\begin{Shaded}
\begin{Highlighting}[]
\BuiltInTok{echo} \StringTok{"Hello Bash!"}  
\BuiltInTok{pwd} \CommentTok{# check working dir}
\FunctionTok{git}\NormalTok{ init }\CommentTok{# initialise git}
\end{Highlighting}
\end{Shaded}

Octave (and MATLAB from the \texttt{RMatlab} package).\\
\href{https://cran.r-project.org/web/packages/R.matlab/index.html}{\texttt{RMatlab\ documentation}}.

\begin{Shaded}
\begin{Highlighting}[]
\NormalTok{b = [}\FloatTok{4}\NormalTok{; }\FloatTok{9}\NormalTok{; }\FloatTok{2}\NormalTok{] }\CommentTok{# Column vector}
\NormalTok{A = [ }\FloatTok{3} \FloatTok{4} \FloatTok{5}\NormalTok{;}
      \FloatTok{1} \FloatTok{3} \FloatTok{1}\NormalTok{;}
      \FloatTok{3} \FloatTok{5} \FloatTok{9}\NormalTok{ ]}
\NormalTok{x = A \textbackslash{} b     }\CommentTok{# Solve the system Ax = b}
\end{Highlighting}
\end{Shaded}

HTML

\begin{Shaded}
\begin{Highlighting}[]
\CommentTok{<!-- links-->}
        \KeywordTok{<div}\OtherTok{ class=}\StringTok{"footer"}\KeywordTok{>}
            \KeywordTok{<a}\OtherTok{ href=}\StringTok{"dd_feed.html"} 
\OtherTok{            class=}\StringTok{"transition fade_in"}\KeywordTok{>}
\NormalTok{                Latest post}
            \KeywordTok{</a>}
            \DecValTok{&nbsp;} \DecValTok{&nbsp;} \DecValTok{&nbsp;}
            \KeywordTok{<a}\OtherTok{ href=}\StringTok{"dd_contact.html"} 
\OtherTok{            class=}\StringTok{"transition fade_in"}\KeywordTok{>}
\NormalTok{                Contact}
            \KeywordTok{</a>}
            \DecValTok{&nbsp;} \DecValTok{&nbsp;} \DecValTok{&nbsp;}
            \KeywordTok{<a}\OtherTok{ href=}\StringTok{"dd_subscribe.html"}
\OtherTok{            class=}\StringTok{"transition fade_in"}\KeywordTok{>}
\NormalTok{                Subscribe}
            \KeywordTok{</a>}
        \KeywordTok{</div>}
\end{Highlighting}
\end{Shaded}

CSS

\begin{Shaded}
\begin{Highlighting}[]
\NormalTok{body }\KeywordTok{\{}
  \KeywordTok{color:} \DataTypeTok{red}\KeywordTok{;}
\KeywordTok{\}}
\end{Highlighting}
\end{Shaded}

Javascript to access \texttt{html} and \texttt{css}

\begin{Shaded}
\begin{Highlighting}[]
\AttributeTok{$}\NormalTok{(}\StringTok{'.title'}\NormalTok{).}\AttributeTok{css}\NormalTok{(}\StringTok{'color'}\OperatorTok{,} \StringTok{'red'}\NormalTok{)}
\end{Highlighting}
\end{Shaded}

Python

\begin{Shaded}
\begin{Highlighting}[]
\NormalTok{x }\OperatorTok{=} \StringTok{'hello, python world!'}
\BuiltInTok{print}\NormalTok{(x.split(}\StringTok{' '}\NormalTok{))}
\end{Highlighting}
\end{Shaded}

Here's a complete list of available languages

\begin{Shaded}
\begin{Highlighting}[]
\KeywordTok{names}\NormalTok{(knitr}\OperatorTok{::}\NormalTok{knit_engines}\OperatorTok{$}\KeywordTok{get}\NormalTok{())}
\end{Highlighting}
\end{Shaded}

\begin{verbatim}
 [1] "awk"       "bash"      "coffee"    "gawk"      "groovy"    "haskell"   "lein"      "mysql"    
 [9] "node"      "octave"    "perl"      "psql"      "Rscript"   "ruby"      "sas"       "scala"    
[17] "sed"       "sh"        "stata"     "zsh"       "highlight" "Rcpp"      "tikz"      "dot"      
[25] "c"         "fortran"   "fortran95" "asy"       "cat"       "asis"      "stan"      "block"    
[33] "block2"    "js"        "css"       "sql"       "go"        "python"    "julia"    
\end{verbatim}

\subsubsection{\texorpdfstring{All from
\texttt{R}!}{All from R!}}\label{all-from-r}

\printbibliography

\end{document}
